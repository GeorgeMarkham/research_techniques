\documentclass[12pt]{article}


\usepackage[]{fullpage}
\usepackage[]{graphics}
\usepackage{natbib}
\usepackage{hyperref}

\title{Topic 22: User identification}
\author{George Markham\\Reg 100130020\\\small \href{mailto:george.markham@uea.ac.uk}{george.markham@uea.ac.uk}}

\begin{document}
	
	\maketitle
	
	\begin{abstract}
	\end{abstract}
	
	\tableofcontents
	\newpage
	\section{Introduction}
	User identification has become increasingly important with the advent of the internet. Businesses, governments, and the general public are sharing personal, sometimes very sensitive information with trusted parties. The key issue being that word \emph{`trusted'}. How does one prove one's identity to another user or system across the globe? 
	
	There are seemingly constantly news stories relating to data theft, last year's Equifax (credit reporting agency) hack could have potentially leaked nearly half the population of the United State's extremely sensitive data \citep{equifax_cnet}. The amount and the sensitivity of the data held on most of the general population of the world can be incredibly damaging if leaked, it can potentially lead to serious fraud that can have very damaging consequences for people. This data theft can be hindered by using secure ways of identifying and authenticating users. Some of those methods will be discussed in section \ref{sect_methods}. If implemented correctly they can potentially mitigate the risk of data theft. Some methods of more secure identification are already being implemented, for example companies such as Google, Facebook and PayPal allow the use of Two Factor authentication, where a user augments the security of their username and password by entering a separate code sent to their device (usually a mobile phone). The code adds extra security as a potential attacker is unable to access a user's account and data unless they also have access to the user's phone. These codes are also resistant to brute force attacks as they are only valid for a short amount of time.
	
	For many years we have relied on passwords to authenticate and identify ourselves but passwords are consistently proven to be insecure and unreliable forms of identification. There are, however, alternatives to the username and password method. Smartphones are including biometric identification methods such as using fingerprints sensors and facial analysis. Other methods for identification can also be considered, for example typing habits, mouse use and speaker recognition can all potentially be used to uniquely identify a person.
	
	
	\subsection{Background and Key Issues}
	The main issue with non-biometric based identification methods is that they are easily forge-able. Take the example of a fingerprint, even amongst identical twins you are still able to be identified with no significant decrease in accuracy \citep{han2004study}. Compare that with the assumption that a password can be known to many different people (either legitimately or illegitimately) then it is clear that biometric identification methods could be superior to traditional username and password methods.
	
	Biometric methods encompass many different ways to identify a user. Sections
	\ref{subsect_biometrics} and \ref{subsect_behavioral_biometrics} will cover the various methods of using both biometric (fingerprints, iris scanning etc...) and behavioural biometric techniques (typing habits, mouse habits etc...) to identify a user. The usefulness of these techniques varies with the use case, for example it is illogical to attempt to use typing habits to identify a user on a telephone call, equally it is not always feasible to use voice identification when authenticating a user for a website.
	
	Alternative methods will also be discussed including the possibility of identifying a user with real-world documents such as a passport, drivers license or ID card. These systems are already secured against fraud so may be a great candidate for use with computer systems. 
	
	In addition to the methods mentioned above there are also public-key encryption algorithms already developed and in use that, although developed for sending encrypted messages, can also potentially be considered forms of identification.
	
	\subsection{Aim and Objectives} 
	This study aims to provide evidence that biometric, behavioural biometric and alternative forms of user identification are superior to a traditional username and password based method.
	To do this the various methods of identification outlined in section \ref{sect_methods} will be evaluated using evidence gathered from previous studies to determine their usefulness compared with each other and username and password identification.
	
	\subsection{Study Plan}
	As part of this study the various methods of identification will be compared and contrasted in order to outline their respective strengths and weaknesses. These comparisons will be based on data from past work external to this study.
	Each technique will be compared to other, similar, techniques and their various merits and weaknesses will be discussed in section \ref{sect_analysis}.
	
	\section{Methods}
	\label{sect_methods}
	\subsection{Usernames \& Passwords}
	Usernames and Password authentication is extremely common. It is the primary way most large websites and apps like Facebook, Google, Instagram etc... authenticate their users, however it has been proven to be rather insecure. The main factor making passwords insecure is that the user must select their own and remember it. This leads users to use the same password across multiple systems, select short, and therefore easy to crack, passwords or to select passwords that are personal to them.
	If a user uses the same password across a number of different systems then if one system is compromised, or stores the password in an insecure manner then the user can potentially have multiple compromised accounts with relatively little effort by malicious parties.
	If a user uses a short password, e.g 4 alphanumreic characters, then it would take a system capable of calculating 1 million hashes per second about 15 seconds to crack that password \citep{kessler1996passwords}. If the user selects a password with just 4 lowercase letters then it could take approximately 0.5 seconds to crack. This is trivial for a malicious party and the kind of computing power needed to achieve this is easily obtainable.
	\subsection{Biometrics}
	\label{subsect_biometrics}
	The use of biometric forms of identification eliminates many of the issues surrounding traditional username and password identification. For example the issues surrounding users selecting weak passwords is eliminated as one does not control one's biometric features and therefore cannot weaken the security of that system. Biometric forms of identification are also more unique and almost impossible to brute force.
	However there are still security issues with biometric forms of identification. One can attempt to trick the sensor for example by creating a fake finger or face \citep{ambalakat2005security}. It is possible to extract finger print patterns from an image taken by a normal digital camera \citep{ogane2017biometric}, however it is also possible to protect against such an attack, researchers created \emph{``BiometricJammer''} to "effectively prevent the illegal acquisition of fingerprints by surreptitious photography" \citep{ogane2017biometric}. The jamming pattern proposed, while effective, must be worn by a person. This means that users must take active action in order to protect themselves. Other methods of biometric identification could be used instead to offer more protection to users.
	Iris recognition provides similar distinctiveness and performance to fingerprint recognition however it may appear more invasive so is unlikely to be accepted as a widespread form of identification. \citep{ambalakat2005security}.
	\subsection{Behavioural Biometrics}
	\label{subsect_behavioral_biometrics}
	Behavioural biometrics pertains to biometric features that are not innate physical attributes. Behavioural biometrics instead focuses on patterns in behaviour, this includes speech and typing habits. Just as physical biometrics can be used to uniquely identify a person through physical attributes so can behavioural biometrics through that person's behaviour. Often this can be done unobtrusively, for example when accessing a website if one could be identified by the keystrokes leading up to that website visit then it eliminates the need for that person to actively prove their identity. By utilising potentially more secure methods of authentication both a user's experience and security can be enhanced.
	\subsection{Alternative Methods}
	Other methods of identification include the use of government verified documents such as drivers licences and passports.
	Trusona Inc. have developed a system that allows users to securely identify themselves with a physical ID card \citep{eisen2017systems}. Trusona has created authentication systems for the FBI \citep{abagnale_2017} thus highlighting how secure this technology can be. However this technology does have a significant barrier to entry, a user needs to have a physical form of identification in order to identify themselves on the system. In the cases of undocumented immigrants or those who do not have documents such as passports, drivers licences or other forms of physical identification it would not be possible for them to use this system.
	
	\section{Analysis and Discussion}
	\label{sect_analysis}
	\textbf{Fingerprints} - Fingerprint identification is widely used in smart phones such as Apple's iPhone and Samsung's S series of devices. Given it's wide use it's important that it is doesn't identify two different people as the same person, measured by \emph{false match rate} (FMR) \citep{delac2004survey}. The FMR of fingerprint identification has been observed as $0.2\%$ \citep{delac2004survey}. This is very low however the experiment only appears to include adults between the age of 20-39, excluding many younger and older people who also need access to fingerprint identification technology. It is not sufficient to include a subset of the population when discussing identification technology. Another issue with fingerprint technology is when user's do not have fingerprints. Fingerprint loss can occur through some medical treatments including chemotherapy. When it comes to smartphone authentication and identification technology it may be prudent to explore alternative biometric methods that are more universal with lower barriers to entry like face identification \citep{prabhakar2003biometric}.
	\\
	\\
	\textbf{Face Identification} - As mentioned above face identification is now being introduced into Apple's new iPhone models (X, XS and XS Max). Traditionally face identification has faced technological issues, specifically a lack of accuracy in "environments with cluttered backgrounds and varied lighting conditions" \citep{prabhakar2003biometric}. Apple's \emph{"FaceID"} technology avoids this issue by projecting 30,000 infared dots onto the user's face to produce a sequence of depth maps and infared images \citep{apple_faceid_2017}. To further enhance security and avoid spoofing the sequences are randomised \citep{apple_faceid_2017}.
	\\
	\\
	\textbf{Handwriting Recognition} - As a method of user identification handwriting recognition is arguably more accessible and useful today as opposed to before the advent of widespread, accurate touchscreen devices. There are two types of handwriting recognition: \emph{on-line} or \emph{dynamic} and \emph{off-line} or \emph{static} \citep{tappert_handwriting}. Off-line handwriting recognition is processed after a person has written something. For example a person may have written a letter on paper, this could then be analysed by a handwriting recognition system and the person's identity could be determined. Because identification does not happen in real-time off-line handwriting recognition is more suited to multi-factor authentication systems, where a user may provide one form of identification initially and then be required to back that initial identity challenge with an example of their handwriting. More useful today, perhaps, is on-line handwriting recognition, where a user may be identified as they are writing using a transducer \citep{tappert_handwriting}. This is of note given the large number of touchscreen devices such as phones, tablets, laptops and even some desktop computers that can be used as inputs to handwriting recognition systems.
	\\
	\\
	\textbf{Signature Recognition} - Quite closely related to handwriting recognition is signature recognition. Similar to handwriting recognition there are two methods of performing the identification, \emph{on-line} and \emph{off-line}. Signatures are already widely used in many aspects of people's lives, this reduces the barrier to entry as most people will already be used to signing documents and have standard way of doing it. It is also worth noting that people can find all aspects of unfamiliar technology quite frightening and by using methods already familiar to people, like signature recognition, it may make user's feel more at ease with the system.
	
	There are issues with signature recognition, people are liable to change their signatures over a period of time, may vary substantially and are liable to forgeries \citep{jain2004introduction}. These are substantial issues when attempting to securely identify a person. Variations make can make systems liable to false matches as two similar signatures with in-built variance may appear to be the same. Signature recognition is also liable to spoof-attacks \citep{jain2004introduction}. Traditional methods of signature forgery will still apply to signature recognition systems and, given that signature forgery has been used to defraud people for many years, it can be assumed that this practice will render signature recognition system insecure for identification over a network.
	\\
	\\
	\textbf{Keystroke Dynamics} - Given that one of the main inputs to most computer systems is a keyboard (either a software keyboard on a touchscreen or a hardware keyboard) it would be useful if one could be identified through it. Using features extracted from a user's typing, including time-features such as down-down times, down-up times, up-down times and the key code of the key typed by the user \citep{typing_auth2005}. These features in the system proposed in \cite{typing_auth2005} gave false rejection rates of $1.45\%$, false acceptance rates of $1.89\%$ for impostor users and false acceptance rates of $3.66\%$ for impostors that had observed legitimate users typing habit. This was proposed as an augmentation to traditional username and password systems to make them more secure without requiring users to provide any further identification or any further effort.
	
	The features required may be difficult to legitimately collect, however. It is possible that software attempting to identify a user based on their typing habits may constitute key-logging which is in breach of many country's laws. It is also likely to suffer from similar issues to signature recognition (discussed above). A person's typing habits are liable to change based on a number of external factors such as fatigue or simply the task one is engaged in. A person's typing will also improve over time as they become more accustomed to typing on a particular keyboard and thus the latency's between key presses that the method above (and other methods such as \cite{shepherd_ibm_keystroke_auth1995}) focused on are liable to change.
	\\
	\\
	\textbf{Speaker Recognition} - Speaker recognition focuses on identifying a user based on their speech. Two main versions of speaker recognition exist: \emph{text-dependent} and \emph{text-independent}. Text-dependent speaker recognition attempts to identify a speaker based on a particular phrase, whereas text-independent speaker recognition attempts to identify the speaker without the need for a specific utterance \citep{microsoft_2006}. In use cases where 
	
	\subsection{Comparison and Contrast}
	
	
	\subsection{Evaluation}
	
	\section{Summary}
	
	\bibliographystyle{agsm}
	\bibliography{references} 
	
\end{document}